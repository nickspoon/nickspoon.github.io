\documentclass{article}
\usepackage[T1]{fontenc}
\usepackage{hyperref, amssymb, amsmath, amsthm, graphicx, subfigure, fullpage}

\usepackage[capitalise,noabbrev]{cleveref}
\usepackage{braket}

\setlength{\parindent}{0em}
\setlength{\parskip}{1em}
\setlength{\topsep}{1em}

% heading ------------------------------------------------------------------------------------------------------------%
\newcommand{\heading}[6]{
  \renewcommand{\thepage}{#1 -- \arabic{page}}
  \noindent
  \begin{center}
  \framebox{
    \vbox{
      \hbox to 5.78in { \textbf{#2} \hfill #3 }
      \vspace{4mm}
      \hbox to 5.78in { {\Large \hfill Lecture #1: #6  \hfill} }
      \vspace{2mm}
      \hbox to 5.78in { \textit{Instructor: #4 \hfill #5} }
    }
  }
  \end{center}
  \vspace*{4mm}
}

\newcommand{\lecture}[4]{\heading{#1}{CS 6832: Quantum Cryptography}{#2}{Nick Spooner}{Scribe: #4}{#3}}

% theorems -----------------------------------------------------------------------------------------------------------%
\theoremstyle{plain}
\newtheorem{theorem}{Theorem}[section]
\newtheorem{conjecture}[theorem]{Conjecture}
\newtheorem{corollary}[theorem]{Corollary}
\newtheorem{fact}[theorem]{Fact}
\newtheorem{lemma}[theorem]{Lemma}
\newtheorem{remark}[theorem]{Remark}
\newtheorem{claim}[theorem]{Claim}

\theoremstyle{definition}
\newtheorem{definition}[theorem]{Definition}
\newtheorem{assumption}[theorem]{Assumption}
\newtheorem{example}[theorem]{Example}
\newtheorem{idea}[theorem]{Idea}
\newtheorem{question}[theorem]{Question}
\newtheorem{observation}[theorem]{Observation}
\newtheorem{proposition}[theorem]{Proposition}

%---------------------------------------------------------------------------------------------------------------------%
% PLEASE MODIFY THESE FIELDS AS APPROPRIATE:
\newcommand{\lecturenum}{1} % lecture number
\newcommand{\lecturedate}{Month Day, 2025} % date of lecture (e.g., 'March 20, 2010')
\newcommand{\lecturetitle}{Title} % lecture title
\newcommand{\scribename}{Werner Heisenberg} % full name of scribe
%---------------------------------------------------------------------------------------------------------------------%

%---------------------------------------------------------------------------------------------------------------------%
% PUT HERE ANY PACKAGES, MACROS, etc., ADDED BY YOU


%---------------------------------------------------------------------------------------------------------------------%

\begin{document}
\lecture{\lecturenum}{\lecturedate}{\lecturetitle}{\scribename}

\section{Introduction}

Write your scribe notes here. Please make sure that you modify the lecture number, lecture title, lecture date, and
scribe name fields in the preamble as appropriate. Also, please write your scribed notes clearly and using good \LaTeX{}
style.

Lorem ipsum dolor sit amet, consectetur adipiscing elit, sed do eiusmod tempor incididunt ut labore et dolore magna
aliqua. Ut enim ad minim veniam, quis nostrud exercitation ullamco laboris nisi ut aliquip ex ea commodo consequat. Duis
aute irure dolor in reprehenderit in voluptate velit esse cillum dolore eu fugiat nulla pariatur. Excepteur sint
aimccaecat cupidatat non proident, sunt in culpa qui officia deserunt mollit anim id est laborum.

\subsection{Theorem Environments}

Here is a demonstration of the theorem environments.
\begin{theorem}
This is a theorem.
\end{theorem}

\begin{lemma}
This is a lemma.
\end{lemma}

\begin{corollary}
This is a corollary.
\end{corollary}

\begin{fact}
This is a fact.
\end{fact}

\begin{remark}
This is a remark.
\end{remark}

\begin{claim}
This is a claim.
\end{claim}

\begin{definition}
This is a definition.
\end{definition}

\begin{observation}
This is an observation.
\end{observation}

\begin{proposition}
This is a proposition.
\end{proposition}

\begin{assumption}
This is an assumption.
\end{assumption}

\begin{question}
This is a question.
\end{question}

\begin{idea}
This is an idea.
\end{idea}

\subsection{Proof Environments}

Here is a demonstration of the proof environments.

\begin{theorem}
\label{thm:thm-label}
This is a theorem with a proof that shows up later in the body.
\end{theorem}

\begin{theorem}
This is a theorem with a proof.
\end{theorem}

\begin{proof}
This is the theorem's proof.
\end{proof}

\begin{proof}[Proof of \cref{thm:thm-label}]
This is the proof of the previous theorem.
\end{proof}


% bibliography -------------------------------------------------------------------------------------------------------%

\bibliographystyle{amsalpha}

\end{document}
